\section*{}
\begin{center}
    {\fontsize{14}{1.5}\selectfont \textbf{CHAPTER II}}\\
    \vspace{12pt}
    {\fontsize{16}{1.5}\selectfont \textbf{Literature Review}}\\
    \vspace{12pt}
    \vspace{12pt}
\end{center}
\setcounter{section}{2}
\setcounter{subsection}{0}
\addcontentsline{toc}{section}{\textbf{CHAPTER II Literature Review}} % Add to ToC
\renewcommand{\theequation}{\thesection.\arabic{equation}}
\renewcommand{\thetable}{\thesection.\arabic{table}}
\renewcommand{\thefigure}{\thesection.\arabic{figure}}
\setcounter{table}{0}
\setcounter{figure}{0}
\setcounter{equation}{0}

\subsection{Literature Review} {

Action detection is a key area in computer vision, it has evolve a lot over the years. Traditional methods mostly focused on trimmed videos where the action of interest was pre-segmented. But now the focus has shift to untrimmed videos, where actions happen along with non-action frames. This change has brought new challenges and chances for researchers.


Early action detection methods used handcrafted features and traditional machine learning techniques. These methods were innovative at the time, but they had problems like view-dependency, handling multiple modalities, and capturing the complexity of human actions. The rise of deep learning has helped  many of these problems creating more robust .



The introduction of convolutional neural networks CNNs was a big breakthrough. Models like IDT encoded with fisher vectors and CNN features showed better performance on datasets like THUMOS'14 and ActivityNet. Multi stage CNN and CDC networks improved temporal action localization by capturing spatial temporal dynamics well.


While many methods focus on offline detection, recent research has tackled online action detection where actions need to be detected in real-time from streaming data. Methods using temporal priors and unsupervised learning have shown potential in this area.


Recent progress includes reinforced learning frameworks and new datasets for online action detection. Future research might look into more advanced models that can handle the high variability and complexity of human actions in real-time scenarios.



Action detection is important in computer vision, especially for applications like surveillance and autonomous systems. Traditional methods depended a lot on handcrafted features and heuristic algorithms, which often did not work well in complex real-world scenarios.


Deep learning has brought significant improvements. Now the backbone of many action detection systems improving accuracy and robustness. Techniques like multi scale sliding window /& spatial temporal parsing have further enhanced detection capabilities.


Online action detection presents unique challenges due to the need for real-time processing and the uncertainty of future frames. They are like the temporal sliding window and frame  end to end framework have been developed to address these challenges.


Recent innovations include using future frame generation and breaking down action classes into temporally ordered subclasses. These approaches help provide more context and improve temporal resolution addressing traditional method limitations.


Benchmark datasets  and activity net have been key in evaluating action detection. These dataset offer various scenarios and actions  allowing researchers to compare their models with state of the art techniques.


Action detection has gone from early heuristic based methods to advanced deep learning models. Initially research focused on segmenting and recognizing actions in pre trimmed videos setting the stage for current methods handling continuous untrimmed video streams.


Early challenges included view-independence, multi-modality, and the variability of human actions. Solutions like genetic algorithms and joint segmentation and recognition addressed some issues but were limited in scalability and generalizability.


Deep learning has revolution action detection Techniques like idt encoded with vectors multi stage CNN and CDC networks have greatly improve the accuracy and robustness of action detection models These methods excel at capturing the complex spatial temporal relationships in video data

Online action detection has make significant progress with methods using temporal priors and unsupervised learning showing improvements Innovations like future frame generation and temporally ordered subclasses provide more context and enhance prediction accuracy in real time scenarios

Future research will likely focus on improving real time capabilities of action detection models making them better at handling diverse and complex actions Creating more comprehensive benchmark datasets and new evaluation metrics will be crucial in advancing this field

These literature reviews offer a comprehensive look at the evolution challenges and trends in action detection highlighting key contributions and future research directions
   
}




% \subsection{Conclusion} {
% The literature review chapter illuminates the research milieu of dominating queries. A review of recent studies indicates that the specific fusion of dominating queries and GPU parallelization remains relatively unexplored. The provided table succinctly outlines recent methodologies, findings, and limitations.
% }